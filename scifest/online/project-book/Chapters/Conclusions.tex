\epigraph{``It is our choices, Harry, that show what we truly are, far more than our abilities."}{Albus Dumbledore}

The previous chapters have discussed the development of an open source implementation to simulating atmospheric dynamics both in the troposphere and the stratosphere on a synoptic scale. Although extremely time consuming, and with a few approximations made along the way, the importance of an open source implementation cannot be understated. It could potentially open up a field to a wider group, and often doesn't receive the media attention it rightfully deserves. 

Future applications of this work are extremely wide ranging, from usage in numerical weather prediction schemes that take in observational data collected from satellites, and ground stations in order to provide a picture of future weather events; to forming at starting point in future research of atmospheric phenomena, ideally reducing the overall research time. While, at the moment, it definitely would be inaccurate to say the software is ready for such usage, with future enhancements which I will touch on in a minute, it most certainly will.

\section{Looking Back}
To prove the hypothesis that `it is possible to create open source software implementation to simulating atmospheric dynamics, with a recurrent neural network and ensemble prediction system being used in combination with a physical model, that such a software implementation has a reasonable execution time to forecast length ratio, and to determine if such an implementation has a statistically significant accuracy improvement over a traditional deterministic physical model', a number of different bechmarking experiments were carried out in the areas of: performance, and accuracy.

\subsection{Performance Benchmark}\label{results_performance}
It is clearly possible to create open source software implementation to simulating atmospheric dynamics, as evident by this report. In regards to performance benchmarking, the ratio between the runtime of the forecast and the fixed length of the forecast was determined. For the physical model, the ratio between the two parameters was $0.0076$; for the physical model with the recurrent neural network enabled, the ratio was $0.0799$; and for the physical model with the ensemble forecast system enabled, the ratio was $0.1147$. This correlates to a usable forecast length of 119 hours (99\%), 110 hours (92\%), and 106 hours (89\%) respectively. Hence, the forecast would be regarded as highly usable, under the assumption that the accuracy of said forecast is also high and also assuming that the forecast is generated on a machine of similar specifications. A reasonably fast internet connection is also assumed due to the large amount of data required for the training of the recurrent neural network. Another point to note is that the runtime of the physical model with the recurrent neural network enabled per time step is rather high. This is could be a potential issue if the recurrent neural network is used in combination with the ensemble forecast system. 

\subsection{Accuracy Benchmark}

In regards to accuracy benchmarking, three distinct benchmarks were carried out. The purpose of the first benchmark was to determine whether the forecast produced by a given scheme of the software was more accurate than the naive forecasting model by using the mean absolute scaled error. Three parameters, geopotential height, zonal wind, and meridional wind, were consistently more accurate than the naive forecasting model across the board. The mean absolute scaled error across the entire forecasting period was $0.9399$, $0.9192$, and $0.9933$ respectively. In regards to the virtual temperature, the physical model and the physical model with the ensemble forecast system under-performed in comparison to the naive forecasting model. The physical model with the RNN enabled, however, performed better than the naive forecasting model with a mean absolute scaled error of $0.9511$. In regards to air temperature, a similar story applies, however, the physical model with the RNN in this case was unable to beat the naive forecasting model, with a mean absolute scaled error of $1.51$. In regards to relative humidity, the physical model with the RNN enabled, and the physical model with the EPS enabled performed dramatically worse than the physical model. The physical model was also unable to match the accuracy of the naive forecasting model with a mean absolute scaled error of $1.17$. That being said, the majority of the forecast schemes within the software had a higher accuracy than the naive forecasting model, hence, the software as a whole has a reasonable level of accuracy.

The purpose of the second benchmark was to determine whether a particular scheme had a statistically significant increase in accuracy over the physical model. In regards to the physical models with the RNN enabled, a statistically significant increase in accuracy was seen in two out of the six atmospheric parameters benchmarked, with a significance level of $0.1$. These were air temperature, and virtual temperature; and had a p-value of approximately $0.0543$ and $0.0115$ respectively. In regards to relative humidity, a statistically significant decrease in performance was observed. In regards to geopotential height and as mentioned previously, a RNN was not developed due to the restrictions created by the COVID-19 pandemic. Hence, a statistically significant increase or decrease in accuracy was not observed. As zonal wind, and meridional wind are determined using geopotential height, an increase or decrease in accuracy was not observed for the aforementioned parameters. Overall, using a recurrent neural network in combination with a physical model, shows significant promise. In regards to the physical models with the EPS enabled, a statistically significant increase or decrease in accuracy was not observed in any of the six atmospheric parameters benchmarked. While an increase in performance was observed across the board, expect in the case of relative humidity, it was not statistically significant. It must be noted, however, that this benchmark doesn't full capture the advantages of using an EPS. An EPS provides a forecast provider with an estimation of the uncertainty in a weather forecast, which is an extremely valuable metric.

The purpose of the third benchmark was to compare the accuracy of the software's most accurate scheme against a proprietary implementation, with the proprietary implementation, OpenWeatherAPI, ultimately being selected. The results shows that there is a statistically significant increase across all of the five atmospheric parameters benchmarked in this scenario. Hence, against a proprietary implementation, the software significantly under-performs. A few key points of information should be noted, however. OpenWeatherAPI does not provide data which changes in the vertical co-ordinate. Secondly, OpenWeatherAPI, as a proprietary implementation, has a significant amount of undesirable properties, as discussed in chapter \ref{5}. Whilst presently, the software does not compete in terms of accuracy, its open source nature will allow for rapid advancements, and improvements in modelling and accuracy. 

\subsection{Sources of Error}

Hence, the hypothesis that was proposed has partially been proven, however, there are a few areas which could have hindered the performance of the software or lead to a possible source of error:

\begin{itemize}
    \item Due to the ongoing nature of the COVID-19 pandemic, schools were required to close in line with guidance from public health experts. This had unintended consequences on my ability to arrange appropriate computational resources. I originally planned on organising said computational resources through the school, however, due to the aforementioned school closures, this was no longer a feasible option to pursue. This limited me to virtual machines provided by Google, Microsoft, and Amazon. While these are viable options if finances support it, they are impractical if said finances are not available. My school, in their eternal generosity, is also the main source of financial support for this project, and as a result of school closures, this limited available finances.
    \item As noted in section \ref{results_performance}, the runtime of the physical model with the recurrent neural network enabled per time step is rather high. This is possibly due to the fact that a GPU was not used in combination with a CPU. Generally speaking, GPUs are fast because they have high-bandwidth memories and hardware that performs floating-point arithmetic at significantly higher rates than conventional CPUs. The GPUs' main task is to perform the calculations needed to render 3D computer graphics. But in 2007 NVIDIA created CUDA. CUDA is a parallel computing platform that provides an API for developers, allowing them to build tools that can make use of GPUs for general-purpose processing. Processing large blocks of data is basically what machine learning is, so GPUs come in handy for machine learning tasks\cite{gpus}. A GPU was ultimately not utilised for financial reasons (it would have cost approximately \$300 a month for the baseline GPU).
    \item The benchmarking process for this particular project began on the 20th of April. It was originally planned to perform the benchmarks on approximately thirty days worth of data, however, due to the combination of restrictions imposed by a lack of computational resources and bugs within the benchmarking code, lead to a reduction in the amount of available benchmarking time. This ultimately lead to the benchmarking process being cut back in order to meet the submission deadline of the 14th of May.   
\end{itemize}

\section{Looking Ahead}
The software is currently in an alpha release state. An alpha version of any software is a very early version of the software that may not contain all of the features that are planned for the final version\cite{alpha}. In this section, I will briefly outline the enhancements and features that will be released in the beta version of the software, which is planned for release in Fall 2020:

\begin{itemize}
    \item In order to ensure that a sufficient amount of computational resources are available for the benchmarking process in the Fall of 2020, I have contacted the Irish Centre for High-End Computing for assistance in this endeavour. They have said they are willing to provide their assistance and expertise to help with this project, including guidance in relation to climate modelling from their climate experts and with assistance in the benchmarking process from one of their computational scientists. A meeting is planned to take place virtually sometime in the next few weeks.
    \item As mentioned in section \ref{implement_rnn}, it is necessary to flatten the 3-dimensional vector that represents the state of the atmosphere in order for it to match the shape of the LSTM cell. While this is an effective approach, a new approach has been shown to be more effective. By combining a convolutional neural network with a neural network, it will result in a more accurate ouput for spatial-temporal type data. A convolutional neural network is a class of deep neural networks. They are also known as shift invariant or space invariant artificial neural networks, based on their shared-weights architecture and translation invariance characteristics. 
    \item In order to improve the simulation of the evolution of temperature and virtual temperature, I would add a diabatic term to the advection equation. The diabatic term, would however, violate quasi-geostrophic considerations. However, diabatic effects can have substantial effects on local atmospheric temperature. For example, the melting of snow falling from a cold layer above into a warm layer below will cause cooling of the warm layer. Diurnal heating plays a major role in temperature changes near the surface. Also, strong ascent can result in significant latent heat release due to condensation of moisture. This warming will oppose the adiabatic cooling due to lift, but net cooling should still occur due to ascent. The less moisture there is in the atmosphere, the more the net cooling will be\cite{describe_quasi}.
\end{itemize}